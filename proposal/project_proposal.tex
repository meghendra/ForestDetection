\documentclass[10pt,letterpaper]{article}
\usepackage[latin1]{inputenc}
\usepackage{amsmath}
\usepackage{amsfonts}
\usepackage{amssymb}
\usepackage{graphicx}
\usepackage[left=1.00in, right=1.00in, top=1.00in, bottom=1.00in]{geometry}
\begin{document}


\section{Describe the data you will analyze}

CropScape and Cropland Data Layer from
United States Department of Agriculture
National Agricultural Statistics Service From 2008 - 2016.

There are 3 main types of (raster) data that we will have:

\begin{enumerate}
	\item National CDL's
	\item National Cultivated Layer (2012 - 2016)
	\item National Frequency Layer
\end{enumerate}

The 2016 Crop Frequency Layer identify crop specific planting frequency and are based on land cover information derived from the 2008 through 2016 CDL's. There are currently four individual crop frequency data layers that represent four major crops: corn, cotton, soybeans, and wheat.

National Confidence Layer

The Confidence Layer "spatially represents the predicted confidence that is associated with that output pixel, based upon the rule(s) that were used to classify it. This is useful in that the user can see the spatial representation of distribution and magnitude of error or confidence for a given classification... This error layer represents a percent confidence associated with each rule and output categorical, classified value. It is expressed as a percentage of confidence. A value of zero would therefore have a low confidence (always wrong), while a value of 100 would have a very high confidence (always right)." For more information on the use of confidence layers please refer to the following paper: Liu, Weiguo, Sucharita Gopal and Curtis E. Woodcock, 2004. Uncertainty and confidence in land cover classification using a hybrid classifier approach, Photogrammetric Engineering \& Remote Sensing, 70(8):963-971. Ultimately, however, the confidence value is not a measure of accuracy for a given pixel but rather a how well it fit within the decision tree ruleset.

\section{Describe at least one question you will answer using your analysis}


\begin{enumerate}
	\item Predict the crops in a particular geographic area in the United States.
	\item Can we utilize and overlay other geogrpahic information to improve our predictions? e.g., Google Earth
\end{enumerate}



\section{Describe the data analytics technique you will use to perform your analysis and why you think this technique will work well for your goal}

\begin{enumerate}
	\item Support Vector Machine
	\item Random Forest
	\item Neural Networks (deep learning)
\end{enumerate}

\section{Describe how you will evaluate whether the data analytics technique is successfully answering the question}

k-fold cross validation?

\section{List some references to existing work that is related or that your project will build on. Related work includes studies using the same data set, using a similar technical approach, or answering similar questions}

Comparison of support vector machine, neural network, and CART algorithms for the land-cover classification using limited training data points
Authors
Yang Shao, Ross S Lunetta
Publication date
2012/6/30
Journal
ISPRS Journal of Photogrammetry and Remote Sensing

\section{List the members of your group.}

\begin{enumerate}
	\item Megh Singh
	\item Arindam Fadikar
	\item Daniel Chen
\end{enumerate}

\end{document}